\documentclass[11pt]{article}
\usepackage{algorithmic}
\usepackage[italian]{babel}
\usepackage[document]{ragged2e}
\usepackage{amsfonts, amssymb, amsmath}

\tolerance=1
\emergencystretch=\maxdimen
\hyphenpenalty=10000
\hbadness=10000


\begin{document}
\begin{titlepage}
    \begin{center}
        \vspace*{1.5cm}
            
        \Huge
        \textbf{MNK-GAME}
            
        \vspace{0.5cm}
        \LARGE
        Relazione
            
        \vspace{1.5cm}
          
        \begin{minipage}[t]{0.47\textwidth}
        \begin{center}
        	{\large{\bf Cheikh Ibrahim $\cdot$ Zaid}}\\
			{\large Matricola: 0000974909}
        \end{center}

		\end{minipage}
		\hfill
		\begin{minipage}[t]{0.47\textwidth}\raggedleft
		\begin{center}
        	{\large{\bf Xia $\cdot$ Tian Cheng}}\\
			{\large Matricola: 0000975129}
        \end{center}
		\end{minipage}  
            
        \vspace{6cm}
            
        Anno accademico\\
        $2020 - 2021$
            
        \vspace{0.8cm}
            
            
        \Large
        Corso di Algoritmi e Strutture Dati\\
        Alma Mater Studiorum $\cdot$ Università di Bologna\\
            
    \end{center}
\end{titlepage}
\pagebreak


\section{Introduzione}
\justify
Il progetto MNK-Game consiste nella realizzazione di un algoritmo in grado di giocare a una versione generalizzata del Tris.\\
La criticità maggiore risiede nella valutazione delle possibili mosse da eseguire che crescono esponenzialmente nel progredire del gioco.\\
Contemporaneamente, l'algoritmo deve avere la capacità di effettuare scelte qualitativamente accettabili.

\section{Scelte progettuali}
\subsection{Interfaccia \texttt{MNKPlayer}}
La funzione \texttt{selectCell} dell'interfaccia \texttt{MNKPlayer} implementa le :)
\subsection{Albero di gioco}
Per la valutazione e la scelta della mossa da eseguire viene implementato un albero di gioco gestito nella classe \texttt{GameTree}.\\
Ciascun nodo dell'albero contiene:
\begin{itemize}		% ----------------- RIDURRE IL PADDING -----------------
	\item La descrizione del nodo rappresentato
	\item Il riferimento al nodo padre e una lista concatenata contenente i figli
	\item Un punteggio euristico
\end{itemize}
\subsection{Euristica sui punteggi}
Tramite la classe \texttt{BoardStatus} è possibile rappresentare ed effettuare la stima del possibile esito di una configurazione di gioco.\\
Tale punteggio euristico viene calcolato tramite un algoritmo basato sulla programmazione dinamica.\\
Siano $M[0 .. n-1]$ il vettore in input contenente la configurazione di gioco di ciascuna cella rappresentata dalle costanti \texttt{PLAYER}, \texttt{OPPONENT}, \texttt{FREE} e $S[0 .. n-1]$ il vettore contenente l'output, ove $S[i]$ contiene una coppia di interi rappresentanti il numero di celle allineabili e il numero di mosse necessarie per vincere selezionando la $i$-esima cella.
\begin{equation}
	S[0] = 
	\begin{cases}
	(0,0) & \mbox{se } M[0]=\mbox{\texttt{OPPONENT}} \\ 
	(1,0) & \mbox{se } M[0]=\mbox{\texttt{PLAYER}} \\
	(1,1) & \mbox{se } M[0]=\mbox{\texttt{FREE}}
	\end{cases}
\end{equation}
% ---------------- DEFINIRE K ----------------
\begin{equation}
	S[i] = 
	\begin{cases}
	(0,0) & \mbox{se } M[i]=\mbox{\texttt{OPPONENT}} \\ 
	(S[i-1].\text{aligned}+1,S[i-1].\text{moves}) & \mbox{se } S[i-1].\text{first} < K \texttt{ AND } M[i] = \texttt{PLAYER}\\
	(S[i-1].\text{aligned}+1,S[i-1].\text{moves}+1) & \mbox{se } S[i-1].\text{first} < K \texttt{ AND } M[i] = \texttt{FREE}\\
	(S[i-1].\text{aligned},S[i-1].\text{moves}-1) & \mbox{se } M[i]=\texttt{PLAYER AND } M[i-K] =\texttt{FREE}\\
	(S[i-1].\text{aligned},S[i-1].\text{moves}) & \mbox{se } M[i]=\texttt{PLAYER AND } M[i-K] =\texttt{PLAYER}\\
	(S[i-1].\text{aligned},S[i-1].\text{moves}) & \mbox{se } M[i]=\texttt{FREE AND } M[i-K] =\texttt{FREE}\\
	(S[i-1].\text{aligned},S[i-1].\text{moves}+1) & \mbox{se } M[i]=\texttt{FREE AND } M[i-K] =\texttt{PLAYER}\\
	\end{cases}
\end{equation}

\subsection{Generazione parziale dell'albero}
\subsection{Valutazione mosse "interessanti"}
\section{Conclusione}


\end{document}

