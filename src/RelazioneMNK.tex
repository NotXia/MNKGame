\documentclass[11pt]{article}
\usepackage{algorithmic}
\usepackage[italian]{babel}
\usepackage[document]{ragged2e}

\tolerance=1
\emergencystretch=\maxdimen
\hyphenpenalty=10000
\hbadness=10000


\begin{document}
\begin{titlepage}
    \begin{center}
        \vspace*{1.5cm}
            
        \Huge
        \textbf{MNK-GAME}
            
        \vspace{0.5cm}
        \LARGE
        Relazione
            
        \vspace{1.5cm}
          
        \begin{minipage}[t]{0.47\textwidth}
        \begin{center}
        	{\large{\bf Cheikh Ibrahim $\cdot$ Zaid}}\\
			{\large Matricola: 0000974909}
        \end{center}

		\end{minipage}
		\hfill
		\begin{minipage}[t]{0.47\textwidth}\raggedleft
		\begin{center}
        	{\large{\bf Xia $\cdot$ Tian Cheng}}\\
			{\large Matricola: 0000975129}
        \end{center}
		\end{minipage}  
            
        \vspace{6cm}
            
        Anno accademico\\
        $2020 - 2021$
            
        \vspace{0.8cm}
            
            
        \Large
        Corso di Algoritmi e Strutture Dati\\
        Alma Mater Studiorum $\cdot$ Università di Bologna\\
            
    \end{center}
\end{titlepage}
\pagebreak


\section{Introduzione}
\justify
Il progetto MNK-Game consiste nella realizzazione di un algoritmo in grado di giocare a una versione generalizzata del Tris.\\
La criticità maggiore risiede nella valutazione delle possibili mosse da eseguire che crescono esponenzialmente nel progredire del gioco.\\
Contemporaneamente, l'algoritmo deve avere la capacità di effettuare scelte qualitativamente accettabili.

\section{Scelte progettuali}
\subsection{Interfaccia \texttt{MNKPlayer}}
\subsection{Albero di gioco}
\subsection{Generazione parziale dell'albero}
\subsection{Euristica sui punteggi}
\subsection{Valutazione mosse "interessanti"}
\section{Conclusione}


\end{document}

