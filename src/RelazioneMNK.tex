\documentclass[11pt]{article}
\usepackage{algorithm2e}
\usepackage[italian]{babel}
\usepackage[document]{ragged2e}
\usepackage{amsfonts, amssymb, amsmath}
\usepackage{mathtools}
\usepackage[margin=3.5cm]{geometry}

\tolerance=1
\emergencystretch=\maxdimen
\hyphenpenalty=10000
\hbadness=10000


\begin{document}
\begin{titlepage}
    \begin{center}
        \vspace*{1.5cm}
            
        \Huge
        \textbf{MNK-GAME}
            
        \vspace{0.5cm}
        \LARGE
        Relazione
            
        \vspace{1.5cm}
          
        \begin{minipage}[t]{0.47\textwidth}
        \begin{center}
        	{\large{\bf Cheikh Ibrahim $\cdot$ Zaid}}\\
			{\large Matricola: 0000974909}
        \end{center}

		\end{minipage}
		\hfill
		\begin{minipage}[t]{0.47\textwidth}\raggedleft
		\begin{center}
        	{\large{\bf Xia $\cdot$ Tian Cheng}}\\
			{\large Matricola: 0000975129}
        \end{center}
		\end{minipage}  
            
        \vspace{6cm}
            
        Anno accademico\\
        $2020 - 2021$
            
        \vspace{0.8cm}
            
            
        \Large
        Corso di Algoritmi e Strutture Dati\\
        Alma Mater Studiorum $\cdot$ Università di Bologna\\
            
    \end{center}
\end{titlepage}
\pagebreak


\section{Introduzione}
\justify
Il progetto MNK-Game consiste nella realizzazione di un algoritmo in grado di giocare a una versione generalizzata del Tris.\\
La criticità maggiore risiede nella valutazione delle possibili mosse da eseguire che crescono esponenzialmente nel progredire del gioco.\\
Contemporaneamente, l'algoritmo deve avere la capacità di effettuare scelte qualitativamente accettabili.

\section{Scelte progettuali}
\subsection{Interfaccia \texttt{MNKPlayer}}
La funzione \texttt{selectCell} dell'interfaccia \texttt{MNKPlayer} implementa le :)
\subsection{Albero di gioco}
Per la valutazione e la scelta della mossa da eseguire viene implementato un albero di gioco gestito nella classe \texttt{GameTree}.\\
Ciascun nodo dell'albero contiene:
\begin{itemize}		% ----------------- RIDURRE IL PADDING -----------------
	\item La descrizione del nodo rappresentato
	\item Il riferimento al nodo padre e una lista concatenata contenente i figli
	\item Un punteggio euristico
\end{itemize}

\subsection{Generazione parziale dell'albero}
Il numero dei nodi generati dell'albero di gioco cresce in modo esponenziale all'aumentare dell'altezza;
per tale ragione è impossibile generare l'intero albero per configurazioni di gioco con un elevato numero di celle.\\
La generazione dell'albero deve essere quindi limitata sia in altezza che nel numero di nodi da elaborare.\\
All'interno della classe \texttt{GameTree} sono quindi presenti le costanti \texttt{MAX\_DEPTH} e \texttt{MIN\_EVAL} rispettivamente l'altezza massima generabile e il numero indicativo di figli di ciascun nodo (variabile in base al numero di scenari favorevoli o sfavorevoli).\\
Al termine di ogni fase di generazione dell'albero, la struttura viene elaborata dall'algoritmo \texttt{AlphaBeta Pruning} per propagare i punteggi delle foglie alla radice ed eventualmente tagliare determinati sottoalberi inconvenienti.\\
Per mantenere costante in altezza l'albero, in seguito alla selezione di una mossa, viene richiamata la funzione \texttt{extendLeaves(Node root)} che estende le foglie dell'albero radicato nel nodo in input.

\subsection{Euristica sui punteggi}
Tramite la classe \texttt{BoardStatus} è possibile rappresentare ed effettuare la stima del possibile esito di una configurazione di gioco.\\
Tale punteggio euristico viene calcolato tramite un algoritmo basato sulla programmazione dinamica che prende in input il vettore $M[0 .. n-1]$ contenente la configurazione di gioco di ciascuna cella rappresentata da \texttt{PLAYER}, \texttt{OPPONENT}, \texttt{FREE} e restituisce il vettore $S[0 .. n-1]$, ove $S[i]$ contiene una coppia di interi rappresentanti il numero di celle allineabili e il numero di mosse necessarie per vincere se si seleziona l'$i$-esima cella. Le seguenti equazioni di ricorrenza descrivono le varie casistiche previste dall'algoritmo:
\begin{equation*}
	S[0] \leftarrow
	\begin{cases}
	(0,0) & \mbox{se } M[0]=\mbox{\texttt{OPPONENT}} \\ 
	(1,0) & \mbox{se } M[0]=\mbox{\texttt{PLAYER}} \\
	(1,1) & \mbox{se } M[0]=\mbox{\texttt{FREE}}
	\end{cases}
\end{equation*}
% ---------------- DEFINIRE K ----------------
\begin{equation*}
\mathclap {
	S[i] \leftarrow
	\begin{cases}
	(0,0) & \mbox{se } M[i]=\mbox{\texttt{OPPONENT}} \\ 
	(S[i-1].\text{aligned}+1,S[i-1].\text{moves}) & \mbox{se } S[i-1].\text{first} < K \texttt{ AND } M[i] = \texttt{PLAYER}\\
	(S[i-1].\text{aligned}+1,S[i-1].\text{moves}+1) & \mbox{se } S[i-1].\text{first} < K \texttt{ AND } M[i] = \texttt{FREE}\\
	\end{cases}
}
\end{equation*}

\begin{equation*}
\mathclap {
	\text{se } M[i]=\texttt{PLAYER}\Rightarrow S[i] \leftarrow
	\begin{cases}
	(S[i-1].\text{aligned},S[i-1].\text{moves}-1) & \mbox{se } M[i-K] =\texttt{FREE}\\
	(S[i-1].\text{aligned},S[i-1].\text{moves}) & \mbox{se } M[i-K] =\texttt{PLAYER}\\
	\end{cases}
}
\end{equation*}

\begin{equation*}
\mathclap {
	\text{se } M[i]=\texttt{FREE}\Rightarrow S[i] \leftarrow
	\begin{cases}
	(S[i-1].\text{aligned},S[i-1].\text{moves}) & \mbox{se } M[i-K] =\texttt{FREE}\\
	(S[i-1].\text{aligned},S[i-1].\text{moves}+1) & \mbox{se } M[i-K] =\texttt{PLAYER}\\
	\end{cases}
}
\end{equation*}
\\
Per trovare il numero di mosse ottimale per ciascuna cella, alla posizione $i$-esima, avviene una fase di propagazione del punteggio:\\

\begin{algorithm*}[H]
\SetAlgoLined
 \For{$j = i-1$ to $i-K+1$}{
 \textit{// L'allineamento viene interrotto da una mossa dell'avversario\\}
  \If{$M[j] = \texttt{OPPONENT}$}{
  	\texttt{break}
  	}
  \textit{// Le mosse precedenti sono migliori, non serve propagare\\}
  \If{$S[j].\text{aligned} = K \texttt{ AND } S[j].\text{moves} < S[i].\text{moves}$}{
   	\texttt{break}
   }
   $S[i-j] = S[i]$
 }
\end{algorithm*}

\subsection{Valutazione mosse "interessanti"}
\section{Conclusione}


\end{document}

